\documentclass{article}



\begin{document}
	\section{Research Projects}

	The author conducted a study to investigate the usage of positive pressure ventilation by UK fire brigades and the obstacles that these brigades have faced in its implementation. He cited a series of studies that had been conducted to study tactical ventilation on the fire ground for the Fire Research and Development Group of the Home Office. One of these, FRDG Report 17/96, An Assessment of the Use of Positive Pressure Ventilation in Domestic Properties, conducted by J.G. Rimen in 1996, determined that fire room visibility increased and fire room temperatures decreased when PPV was used as compared to when it was not used. It also offered guidelines for the placement of the fan and other considerations. Another, this time conducted by M. Thomas in 1998, FRDG Report 8/98, Measurements of the Firefighting Environment Made during Tyre and Wear Metropolitan fire Brigade’s Positive Pressure Ventilation Trials at the Fire Service College, looked at the use of positive pressure ventilation when the victim was between the fire and the exhaust point, which some companies had expressed concern about because their thought was that the fire would be burning more intensely towards victims laying in this path. The study found that, “overall the use of PPV would cause less harm to the casualty than the fire itself would already have done by the time the firefighting commenced” (Thomas 1998). Furthermore, he stated that there was no major difference in fire spread between the use of natural ventilation and PPV. A third report, conducted by A. Hay, Positive Pressure Ventilation- A Study of Overseas Experiences, says that problems arising from the use of a positive pressure attack are more often from misuse of the tactic arising from a lack of training on the subject and not from any fundamental problem with the tactic itself.\cite{YatestheU}\\
	
	\indent The University of Central Lancashire, in conjunction with the Lancashire Fire and Rescue Service, conducted a series of experiments to look at the use of PPV in residential houses. They conducted these tests in a row of abandoned houses in Preston, UK. In the tests, it was observed that flashover was reached in most cases. When the wrong window (one that was not connected to the fire room) was broken, the fire was observed to be “dragged” so that the flow path included that window. In a “successful” pattern, where the window was broken in the correct place, it was observed that the fire conditions in the fire compartment intensified, while they improved throughout the rest of the house. Furthermore, conditions distant from the fire room were considered tenable according to gas measurements within seconds of application of PPV. The growth of the fire was not greatly increased by PPV, but burning increase was seen under some circumstances because the fire spread when PPV is applied is horizontal. The study found that the key to a successful positive pressure attack is the correct choice of a vent location, so that the fire is not moved from its original location.\cite{StottApplyingpressure}\\
	
	\indent The Tyne and Wear Metropolitan Fire Brigade conducted a series of tests to compliment the Fire Research and Development Group 8/98 publication. The tests involved a three bedroom house with a realistic fire load. In each test a mannequin was placed between the fire and the exhaust point. The dummy was outfitted with polymer skin, to represent human skin, in addition to instrumentation to measure heat flux, CO concentration, gas temperature, and pressure.  The tests showed that the PPV reduced the heat flux, which was 3.1~kW/m$^{2}$ before PPV was applied, by 67\% in 30 seconds and reduced the CO level by 65\% in 60 seconds.\cite{TTurpinBowserNatPerspective}\\
	
	\indent. A study was intended to determine the effectiveness of different ventilation methods on a large 39m x 11.2m building with 8.9m ceiling height. The fuel load was a 2 $m^{2}$ pool of methanol, which was supplemented by propylene glycol smoke machines that were suspended 3m above the fire source.  The building had two doors and two windows. The study concluded that positive pressure ventilation increased flow of hot smoke through an outlet, and also created turbulence. This turbulence disturbs the thermal layer, forcing hotter air down into an area that was previously clear and cooler. Larger fans also increase both fire spread and burning rate. According to the author, this poses a threat to firefighters and victims alike.\cite{SvenssonFireVentinLargeFireHall}\\
	
	\indent The authors conducted a series of experiments in acquired structures in conjunction with the University of Central Florida and the Orange County Fire Rescue Division. The purpose was to develop an underwater simulation of live fire tests to compliment firefighters’ understanding of PPV. Two tests were conducted: One to examine PPV use with an attic fire and the other to examine the effects of PPV on an interior room with the only source of ventilation being the inlet door (no exhaust point made). The study showed that for both tests, the use of PPV significantly lowered temperatures and toxic gas levels and raised visibility. Furthermore, they determined that a fire attack using positive pressure attack techniques could be conducted more quickly than a conventional attack without the use of PPV. The PPA was usually conducted in 5-6 minutes while the non-PPA attack took 9-10 minutes. Also, when PPV was used with an attic fire, smoke was cleared from the lower regions of the house and from the attic. The second test, of the interior room with only one ventilation point, demonstrated the need for careful placement of PPV fans, because the fan actually worsened visibility conditions, since it was blowing products of combustion back into the house. The authors maintained that while visibility was worsened, temperatures were still improved in this test.\cite{AdvancesinPPV}\\
	
	\indent The University of Le Havre commissioned a research project in cooperation with UK fire brigades to begin the groundwork for a European standard on PPV. Most of the research related to the technical considerations of the PPV fans themselves. They discovered that it is impossible to get a true seal on the door, as there will always be a neutral pressure plane at the door frame. They also noticed the turbo and conventional units had the same performance as far as size of the air cone, but the turbo unit resulted in increased air flow in the structure.\cite{LeHavrePPV}\\
	
	\indent The Lancashire Fire and Rescue Services and the University of Central Lancashire conducted a series of experiments to further previous studies on the use of positive pressure ventilation in residential houses. They conducted 9 experiments, which examined different locations of exhaust holes and also compared positive pressure attack to the use of post-knockdown PPV. The results showed that fire growth was not greatly increased by the application of PPV, even under ventilation limited conditions. The research noted that larger flames were sometimes seen out of the exhaust vent during the application of the PPV fan, although this occurred simultaneously with flashover, and may have been attributed to that. During one of the experiments, the crews experienced a “flashback,” where flames briefly rolled back down the hallway at them, despite the application of PPV. This was attributed to having too small of an exhaust hole. The study also determined that the original ratio between inlet and outlet, which was 2:1, selected based on previous research, was too small. In addition, they confirmed that poor choice in vent placement could draw the fire into unburned parts of the structure. The study found that fan placement was not as important, probably because the capacity of the fans was large compared to the size of the structure. With regards to post-fire PPV, they found that it very effectively improved conditions, and that rekindling was not an issue as long as the fire was adequately extinguished.\cite{StottPreston}\\
	
	\indent . A study was carried out to determine if the use of positive pressure fire attack would have a negative effect on a victim located between the seat of the fire and the exhaust point. The fires were conducted in a concrete burn building and the fire loading was pallets, straw, paper, and kindling. The overall conclusions that the study reached were that the detrimental effects that the victim between the vent and the exhaust were negligible, for both natural ventilation and PPV, because any burns or smoke inhalation would have occurred prior to fire department arrival. It was also noted that PPV did not noticeably increase fire intensity or growth, with the exception of some local flaming, which was a result of the entrainment of additional air into oxygen-starved combustion products. PPV always significantly improved visibility in the structure pre- and post-fire. When using PPV compared to natural ventilation, the test results showed that fires were easier to extinguish and resulted in less smoke and water damage. The study also compared different sized fans (27”, 18”, and 21”) and found that speed of heat and smoke removal is directly proportional to fan size.\cite{BowserFireServiceCollege}\\
	
	\indent Dr. Martin Thomas, head of the Fire Experimental Unit of the Home Office, describes the conclusions that had been drawn from the FRDG studies on positive pressure ventilation. Through a series of tests conducted largely in concrete burn structures, the group reached a number of conclusions about the practical use of PPV by UK fire brigades. Chief among these considerations are that wind dominates PPV and that a positive pressure attack must be delayed until the fire compartment has been identified. If the fire compartment is not known and the wrong room is vented, fire is likely to spread to that compartment. The study also found that if the wind is in line with the inlet hole (it is working with the PPV), then the fan is unnecessary because the wind overwhelms it. Conversely, if the wind is in line with the outlet vent, it may overcome the positive pressure attack, forcing fire onto the advancing attack line. When used correctly, however, PPV drastically improved conditions for the attack company, with only a small increase in fire spread due to increased air entrainment. Other considerations that the studies included are that the fan should be positioned so that the cone of air covers the door, making for an even air flow throughout; that PPV works better in smaller compartments, making little to no difference when a fan is used to ventilate a larger volume; and that a PPV can very effectively clear a cellar if an exhaust vent can be made. The group also conducted one experiment where, instead of a fan they used hydraulic ventilation, directing a fog stream into the door. They noted that this method was almost as effective as the fan, but entrained 450 liters of water into the building.\cite{HomeOfficeRes}\\
	
	\indent A study was undertaken to examine the characteristics of the jet of a PPV fan. The first part looked at the normal jets generated by an axial turbine. A second part looked at these jets once again under more realistic conditions, with a wall and floor in place. The author concluded that the analysis of a jet produced by a PPV van must be conducted under conditions that represent real conditions, where the fan is blowing into a compartment with an inlet door and a smaller exhaust hole, and then the pressure inside the compartment is measured.\cite{AxialTurbine}\\
	
	\indent While the Home Office has conducted testing on the merits of positive pressure ventilation as a means of fire attack, their resources are limited and it is largely up to the individual brigades themselves to push the matter forward. The Lancashire Fire and Rescue Brigade acquired a series of one story structures with realistic fire loading to conduct tests on. The tests showed positive results, although the author notes that these tests highlight the need for effective training and another series of tests are needed on two story residential structures to further examine the issue.\cite{StottSingleStory}\\
	
	\indent The Mechanical Engineering Department at UT-Austin and the Austin Fire Department conducted tests to analyze the flow paths created by positive pressure attack. They had a series of fire-hardened acquired structures with a fuel loading of polyurethane mats over a burner. The test involved a “victim room,” which was next to the fire compartment. The “victim” was instrumented with heat flux gauges and TC’s.  The results showed that when the window of the victim room was opened, there was more cooling at higher levels with PPV than with natural ventilation, but there was a slight increase in temperatures at floor level due to mixing of the fresh air with gases. When the fire compartment was vented, there was less of a mixing effect with the victim room.\cite{UTAustinPPV}\\
	
	\indent  In a study to examine the thermal effects of positive pressure attack downwind of a fire, AFD and UT-Austin conducted 20 burns in a house on the outskirts of Austin. Two ventilation patterns were selected, one where the fire compartment was vented, and another where the room adjacent to the fire compartment, the victim room, was ventilated. The house was protected with double layer gypsum and all rooms out of the flow path were sealed. HRR was approximately 2.4 MW. The results showed that, in general, positive pressure ventilation was quicker and more effective than natural ventilation. The decrease in temperature was significant at higher levels, but less than 10\% at victim height, due to the turbulence caused by the PPV fan. This turbulent mixing of gases could be seen by the IR cameras. The study also found that with PPV, victim room temperatures decreased more with fire room venting than with victim room venting. Both natural and positive pressure ventilation were found to increase visibility, although the increase was noted much more quickly with PPV. \cite{Lakshmin}\\
	
	\indent NIST conducted a study to find effective techniques for PPV for ventilation and smoke control in high-rises. They conducted live fire experiments in Chicago, New York, and Toledo. The conclusions from these tests were that PPV fans should be placed 4-6 feet back from the door and angled up at 5-15 degrees; that a v shape is more effective than a series; that fans at the base of the stairs alone will not be effective, and that the CO produced by the fans is not of major concern.\cite{KerberWorldSafety}\\
	
	\indent The purpose of this study was to take a look at methods used by the Swedish Rescue Services to combat fire in residential occupancies, specifically positive pressure ventilation. The tests were constructed in a fire-resistant building with three rooms, 2 hatches (small openings, likely for water drainage), 4 doors, and a window. The fuel load was a .5 m2 heptane burner, which sat on a load cell to determine fuel loss rate. In the tests, by the time PPV was applied, the fire had become ventilation limited. Once PPV was applied, the flames increased, indicating that the burning rate had increased. If firefighting operations were delayed, then this increased burning rate may cause uncontrolled fire spread. The report concluded that the use of PPV in the fire attack increased the mass loss rate of the fuel, increasing the burning rate. It also drastically improved working conditions for the fire department, allowing a quicker fire attack.\cite{SvenssonVentFFOps}\\
	
	\indent The Technical Research Centre of Finland conducted a study to compare tactical ventilation techniques in firefighting in domestic occupancies. They did this by constructing a ¼ scale building with three rooms, an entrance room, a target room, and a fire room. The target room had a small window, the fire room had a small window and a large window, and the entrance room had a door. The walls were made of Siporex tiles and the ceiling was Gyproc board. The fire load was a propane burner beneath a 24 stick wood crib and plywood wall corners, to precipitate rapid fire growth and flashover times. After 40 tests, the results showed that the best results were when window vent was in the fire room and not the target room. Both natural and positive pressure ventilation improved conditions, with PPV improving them faster. With PPV, the best conditions were achieved when the fan was directed into the entrance room, so as to pressurize it, because when the fan was directed into the fire room, visibility was obscured by the turbulence of the smoke layer. The study recommended a hole size from ¾ to 1 ¼ the size of the inlet, with better conditions being observed with a larger hole that was closer to the fire. The optimal PPV flow rate was noted to be 128-192m3/min in a full scale setting. When PPV was used with a correct ventilation pattern, temperatures in rooms remote from the fire room dropped, visibility improved, and fire room temperatures increased. The study also examined the possibility of a backdraft being initiated, and while the results were inconclusive, it noted that there would be a delay before any “fireball” was formed. The study also advises that if anything goes wrong, all the firefighters need to do is simply turn off the fan.\cite{TuomisaariPPV}\\
	
	\indent As a result of a lack of objective data regarding the use of positive pressure ventilation in a fire attack role, the Fire and Rescue Services Division of the North Carolina Department of Insurance did a study to examine the effects of three different ventilation scenarios in a masonry burn building with a fuel load of diesel fuel and oak pallets. The results showed that using PPV in a positive pressure attack role reduced the duration of elevated CO levels more than either of the other two forms of ventilation. Also, PPA was the only form of ventilation that prevented the accumulation of carbon monoxide in the floor above the fire floor. There was no evidence that PPV forced products of combustion into the upper level. PPV improved visibility and reduced the time of exposure to toxic gases throughout the structure. The study did state, however, that it was evident that the use of PPV in a pre-attack setting “accelerated the fire.” The authors maintained that although fire growth was accelerated, access to the fire was quicker because of improved visibility conditions.\cite{GiguereToxicGas}\\
	
	\indent The Tyne and Wear Fire Brigade conducted a series of three trials in a single day to examine the merits of PPV before and after a fire attack, and compared them to natural ventilation. The fire compartment was a three bedroom house, with the living room being the room of ignition. The walls and ceiling were protected with fiber reinforced gypsum. The fire load was two armchairs, a sofa, a chipwood coffee table, a polypropylene carpet, and polyester curtains. The first test examined PPV in a pre-attack application after the fire had gone into a smoldering decay phase, the second examined fire suppression without the use of PPV, and the third examined pre-attack PPV applied at the onset of flashover in the compartment. It was noted by the author that although the same fuel load was used for each test, the rate of fire growth was different for each test. The results showed that showed that when tests 1 and 2 were compared, the pre-attack PPV resulted in a 67\% reduction in heat flux at the victim location at 30 seconds after application. The same reduction was not seen in Test 2 until 50 seconds after intervention. PPV also decreased CO concentration by 65\% in 60 seconds. A significant temperature reduction was not noted in test 2, without PPV, until 30 seconds after suppression. With PPV, a reduction was noted almost immediately. In test 3, unlike the other test, timber panels were applied to the walls to accelerate fire growth. The oxygen concentration was at 8\% in the fire room at the time of PPV application. At first, temperatures were reduced because the hot gases were cleared from the compartment. Within 30 seconds, however, the air flow had raised the oxygen concentration enough that the pyrolyzed gases from the wood ignited, rapidly raising the temperature and bringing about flashover in the fire room. Despite the flashover conditions present, however, the pressure caused by the PPV prevented the flames from entering into the hallway, even though the front window had been opened. The study concluded that although in some cases PPV will improve conditions at the victim location (between the fire and the exhaust vent), as was the case in the comparison between tests 1 and 2, this is not always the case, as was evident with the flashover conditions seen in test 3.\cite{Tyne&Wearauthorless}\\
	
	\indent This study uses Computational Fluid Dynamics, specifically the CFD code Simulation Of Fires In Enclosures (SOFIE), to examine the effect of different firefighting tactics on backdraft scenarios. The five situations examined were a reference scenario where the firefighters simply opened the door and stood to the side, an offensive attack scenario where the firefighters are kneeling in the doorway and blocking part of the flow path, a defensive scenario where the firefighters do not make entry and break the window and open the door, an offensive attack with PPV and a properly placed exhaust vent (Close to the seat of the fire), and an offensive attack with no exhaust vent. The results showed that in the second scenario (offensive with no PPV), it only takes a few seconds for the gravity current to create a flammable region large enough to pose a problem for the fire attack crews. It also showed that closing the door to the width of the hoseline after entry does not reduce the chance of a backdraft scenario. The third scenario (defensive firefighting) showed that the more vents that were opened, the quicker the apartment was vented. This assists in finding the seat of the fire and results in quicker extinguishment, according to the authors. The fourth test, with a properly implemented positive pressure attack, showed that while initially the fan causes greater turbulence in the apartment, meaning more mixing of gases and a greater chance of backdraft, but this increased backdraft possibility is short-lived, because the heat and gases are quickly forced out of the fire compartment by the fan. The authors state that if used correctly, it can quickly improve conditions in the compartment, but firefighters should keep the possibility of ignition of the flammable layer in mind and protect the exposures out of the vent opening. The final test, which simulated a PPV attack with no exhaust opening, showed that if the gravity current is allowed to reach the back wall of the apartment, the chance of backdraft was greatly increased. Also, it showed that the fan could push fire gases into previously uninvolved parts of the structure, which could cause additional unnecessary property damage. Overall, the authors recommend that fire attack crews make sure to cool down the hot gas layer before making entry, to reduce the possibility of a backdraft. They also state that PPV can be used to help salvage victims, but there must be a clear flow path between the fire and the exhaust port. They advocate a balanced risk assessment method, saying that if the structure is unoccupied, a defensive operation with horizontal ventilation is the ideal tactic for locating and suppressing the fire.\cite{Backdraft}\\
	
	\indent This study was undertaken to examine the effectiveness of three types of PPV fan (an 18” 5.5 HP gas-powered fan, a 24” 5.5 HP gas-powered fan, and a 35” water turbine fan), in two different test structures. The first test structure was 800 m3 and the second was 420 m3. All of the tests were conducted without any fire or smoke involved, since only the rate of volume flow and the pressure were being analyzed. The volume flow rate was measured with bi-directional velocity probes and the pressure was measured with a static pressure sensor that measured the difference between the pressure inside of the test building and atmospheric pressure. The variables included the size of the inlet and exhaust openings, the distance between the fan and the inlet, the type of fan, the arrangement of fans (series, parallel, etc.), and the size of the building. Theoretically, the report states, QF, the volume flow rate through the exhaust opening is independent of the volume of the container. It does state, however, that the air change rate, QF/V, is dependent on the gross volume, and this is the parameter that effects how efficient PPV is at clearing smoke. The results of the study showed that each fan achieved its peak flow rate at different distances from the doorway, but the graph was relatively flat between distances of 1 and 3 meters, meaning the distance from the doorway was not particularly critical. It was also found that the farther back the fan was placed, the greater the maximum volume flow rate as compared to the primary flow rate out of the fan itself. This is likely due to the additional entrained air. Also, the peak volume flows are seen with an inlet/outlet ratio where the outlet is twice as large as the inlet. The study also makes a point that if the goal is to pressurize an area to prevent smoke movement, a small inlet and smaller outlet are preferred. When comparing fans in series and in parallel, the study showed that putting the fans in series had little effect, and that when the fans were placed in parallel, both volume flow and pressure in the compartment were increased.\cite{PPVmediumhouse}\\
	
	\section{Line of Duty Deaths and Close Calls}
	
	\indent A Career Firefighter in Florida was killed when he became separated from the rest of his company while performing a search off of the line. After arriving on scene, the Incident Commander decided on a Positive Pressure Attack and had the Ladder company set up a PPV fan at the door while the victim’s company made entry with a hoseline. After moving down a hallway, interior conditions rapidly started to deteriorate, and the company officer ordered the three man company to evacuate the building. When the officer exited, the victim was no longer behind him and fire was burning out of the front door.  After an extensive search, the victim was found in the kitchen and was pronounced dead on the scene. The autopsy revealed that he had succumbed to Carbon Monoxide poisoning. His PASS device had melted, and was not sounding when the victim was found.\cite{NIOSHF2000_44}\\
	
	\indent A Positive Pressure Attack was applied on this 100 ft. by 100 ft. auto salvage building, which had light smoke showing on arrival. The Incident Commander opened a roll-up door and instructed one of his attack companies to set up a PPV fan at the front door. He additionally asked the owner of the structure to tear down part of the metal wall opposite the front door with the forklift. After the ventilation was accomplished, two companies made entry with 1.5” attack lines to find the seat of the fire. Interior conditions were originally reported as having a layer of thick black smoke at four feet off of the ground with about 5 feet of visibility. It was reported that conditions were good and the PPV was working. The interior companies initially had difficulty finding the seat of the fire. Once the seat of the fire was located and extinguished, command ordered the two attack companies out with the intent of starting mop-up operations. While one of the attack companies was backing their line out, they were hit by a sudden blast of thick black smoke and heat and were disoriented. The Chief that was manning the nozzle on this line was unable to back out and succumbed to carbon monoxide poisoning. The second victim ran out of air while searching for the first victim and also died of CO poisoning.\cite{NIOSHF1998_32}\\
	
	\indent In a situation where a tower company was responding mutual aid for a fire in another district, two firefighters were injured, one severely. When the tower company arrived on scene, there was heavy fire on the first floor, which had spread to the second floor and the cockloft. The building was a type III ordinary construction. The officer noticed that a PPV fan was blowing into the front door, which he shut off before his company was sent to the roof. The officer stated that he fell through a hole in the roof, and when his colleagues pulled him out, he saw that 10-15 foot flames were coming through the vent in the roof. The author also stated that while the PPV van was in use, the picture window in the front of the house had been broken.\cite{GoldfelderCritCoord}\\
	
	\indent A truck company was en route to do a business inspection when they passed by a house with heavy black smoke showing. They radioed dispatch, who sent additional units to the scene. They pulled a 1 ¾” crosslay and commenced a positive pressure attack. The family room that they entered into had 18-20 foot vaulted ceilings, and smoke was almost to the floor level, limiting visibility to zero. The attack team made entry into the building. The nozzle firefighter made entry into the fire room without realizing it, and had flames exiting the room above his head. Upon exiting the structure, the nozzleman’s helmet was charred and his nomex hood was unusable. The lesson to take away was to slow down and wait for PPV to take effect before making entry to the building.\cite{VaultedCeilingsFFCC}\\
	
	\indent A basement fire in a balloon frame structure led to the death of a volunteer firefighter in Massachusetts. The victim, another firefighter, and a deputy chief were in the basement and reported that they had a knock on the fire. The DC ordered the truck company to set up a PPV fan at the front door and vent the basement windows. Shortly after application of PPV, thick black smoke filled the basement, second, and third floors, and visibility in the basement dropped to zero.  Heat began to intensify dramatically. The DC called a mayday and ordered the two firefighters to get out of the house. The DC ran out of air and had to be helped out of the house, and after a roll call it was determined the victim was missing. An engine company attempted to make its way downstairs to help the victim, who was unresponsive, but the basement was engulfed in flames. When they exited the structure, they had to remove their gear which had ignited as a result of the extreme heat in the basement. The investigation conducted by NIOSH determined that the firefighters were located between the seat of the fire and the exhaust opening when PPV was applied. Among their recommendations that resulted from the investigation was that ventilation should be coordinated with fire attack.\cite{NIOSHF2004-02}\\
	
	\section{SOP’s/SOG’s}
	
	\indent Phoenix Fire Department’s SOPs describe that PPA is useful improving victim tenability by removing hot and toxic gases from a structure and decreasing property damage by reducing smoke damage. It also eliminates the need for potentially dangerous roof ventilation. In order for PPA to be an option for the attack companies, there must be a viable exhaust opening in the fire area and the positive pressure must be applied from the unburned side of the fire. Precautions for this tactic include: protecting exposures due to the possible “blowtorch” effect, where flames shoot far out of the window; not to make too many exhaust opening and not to put these exhaust opening in the wrong place, because these will decrease the effectiveness of PPA and may spread the fire into unburned areas; checking void spaces for extension; and not to direct a hose stream into the exhaust opening.\cite{PhoenixPPVSOG}\\
	
	\indent. This sample SOG laid out by Kriss Garcia provides a template for fire departments to adopt in order to implement a positive pressure attack strategy. The document outlines four steps for a first arriving company to start a positive pressure attack. First, a vent hole location should be identified and the blower should be positioned. Crewmembers should be designated to carry out each task, and the blower should be positioned 6-10 feet from the door. The fan should be started as soon as possible. Next, the officer should survey the exterior and create or improve a vent opening. The vent opening size depends on the type of structure being ventilated, but should usually be 2-3 times the size of the inlet (usually 2-3 windows for a residential structure).  The third step is to begin pressurization and commence fire attack. The blower should only be started once an exhaust is made and an attack line is charged and in place. Attack crews should wait about 30 seconds before entering the structure I order to allow pressurization to take place. Finally, fire companies should aggressively overhaul to ensure that the fire is completely extinguished. It would be wise to turn off PPV blowers for a short period to ensure that there are no wisps of smoke that the blower may be hiding. Additional considerations that the SOG includes are exposure protection, where a PPV fan can be used to pressurize an adjacent house to prevent fire extension, and in basement fires, where caution must be taken to provide an adequate exhaust while keeping the vent hole as close to the seat of the fire as possible.\cite{SuggestedSOP}\\
	
	\section{Fire Service Publications}
	
	\indent Carlson discusses some concerns that readers had voiced about some of the possible consequences of a Positive Pressure Attack. First, he addresses the concern that PPA will force smoke and heat into habitable apartments, as opposed to venting these toxic gases out of the structure. Carlson says that if the exhaust point is correctly placed, the smoke and gases should follow the path of least resistance, which would be out of the hallway and through the exhaust point. Similarly, he argues that Positive Pressure Ventilation should not disrupt the thermal balance. He says that rather than forcing the hot air downward, PPV will force it out of the structure. He also acknowledges the concern that forcing air into a structure will entrain oxygen into materials that may be smoldering in a wall cavity or a similar void space, leading to rapid fire growth. While Carlson states that this is concern, he also emphasizes that it is for this reason that the introduction of PPV must be coordinated with a swift and effective fire attack, so that excess air is not entrained into hidden fires, complicating the fire attack.\cite{CarlsonFE}\\
	
	\indent Campbell identifies two problems that are responsible for the hesitation of many fire departments in the United States to adopt the positive pressure attack tactic: the tradition of the fire service, which has been reluctant to accept radically new technology in the past, and the hesitation of many firefighters to force copious amount of fresh air into an under ventilated structure. He maintained that while difficult to conceive, there was a large amount of anecdotal evidence suggesting the validity of the technique. He listed four important considerations that a fire officer had to keep in mind when implementing a positive pressure attack. The first of these was completely sealing the entrance with an air cone, which he states will keep products of combustion from burning towards the door. Second, there must not be any extra exhaust points, such that the flow between the front door and the exhaust opening will be interrupted. His third consideration was the size of the exhaust hole to be used, which he maintained was related to the size and capacity of the blowers being used. His final consideration was that every firefighter must have enough training on the subject to perform their task effectively, and that officer must communicate so that the attack is coordinated. Contraindications to a PPV attack include backdraft or explosive atmosphere conditions, and a situation where victims or firefighters are between the exhaust and the fire.\cite{traditionvtechnology}\\

	\indent Because of the changing fire environment, US fire departments must reexamine the tactics that they use, specifically their ventilation tactics. Kriska recommends a positive pressure attack, but acknowledges that such an attack is not applicable for every situation. “PPA must be used correctly. The seat of the fire must be identified, an exhaust opening made in that area. Period.” He also identifies a series of “musts” for positive pressure attacks, including using the incident command system, having a written operational document, being properly trained, knowing exactly where the seat of the fire is, having an exhaust opening in the area of the seat of the fire, and having a inlet opening remote from the exhaust. He also states that it is essential to look for backdraft indicators and to start the PPV fans before the attack crews enter the building. He emphasizes that properly used PPA can reduce the chances of firefighter injury.\cite{KriskaPPA}\\

	\indent Positive pressure attack is a necessary change for a safer fire ground for both victims and for firefighters. The authors say that PPA is a solution that clears the atmosphere to help victim tenability while at the same time clearing the way for the attack crew, hastening the fire attack. The authors go on to address some of the concern that many firefighters have expressed with the idea of a positive pressure attack. Citing field experiments that they have conducted in makeshift house with pallets and straw as a fuel load, the authors say that fire will not be pushed into other compartments, and will not be pushed from void spaces to the attic. They stated that the only time that fire was observed to spread to the attic was when a hole was cut for vertical ventilation. This is because the fire is seeking the path of least resistance, which is straight out the exhaust point. When addressing the concerns that a PPA may require extra staffing or time, the authors say that a well-trained crew can conduct PPA in the same time that it takes for a conventional attack, while eliminating the need for vertical ventilation. With regards to time, they say that the victim’s best chance for survival is immediate removal of toxic gases from the structure. When talking about where to put exhaust openings, the best place is as close to the seat of the fire as possible. According to the authors, however, you cannot make an exhaust opening in the wrong location; some exhaust locations are simply better than others. They go on to say that if there is insufficient pressure coming out of the first exhaust opening that you make, you should take another window, making sure that no crews are inside of the structure while you do this. The blowers should be started ten seconds before the initial attack crew makes entry. The authors also argue that it is impossible to place the blower too close to the seat of the fire, because fire will not be blown throughout the structure.\cite{PositiveReinforcement}\\

	\indent The advantages of PPA over natural ventilation are that improves conditions faster, requires less manpower, and decrease property damage from smoke damage and fire spread. PPA also decreases the risk of flashover, because smoke and hot gases are being forced out of the building. When looking for the appropriate blower technology, the fan should be gas powered for easy deployment, should be shrouded to increase entrained air and the exiting CFMs, should be adjustable so that the angle can be changed accordingly, should have pneumatic tires so that on firefighter can operate it quickly, and it should have the fewest possible number of switches . All of these facets should ensure an easy-to-operate and easily deployed PPV fan.\cite{powerofPPV}\\

	\indent Positive Pressure Attack can be a useful tool to quickly remove toxic gases from a fire environment, but cannot be used in all situations. The author advocates the use of the acronym LELO before using PPA. LELO stands for Life (Are all victims and firefighters out of the building and out of the potentially bad parts of the flow path), Explosion (Is there evidence of backdraft or dust explosion conditions), Locate (The seat of the fire must be identified, or efforts at PPV will be futile), and OK (If all of the previous conditions are met, it is OK to use PPA). He also advocates the use of PPV and a dry chemical fire extinguisher for chimney fires. This author’s stance on PPV is that all victims must be located before it is ever applied, lest they be between the fire and the exhaust.\cite{TurgeonPPV}\\

	%\indent This article presents a set of guidelines for the use of positive pressure ventilation, both pre- and post- attack. The author states that the vents should be made before PPV is started, which will prevent the uncontrolled spread of fire by pressurizing the house before a vent location is made. When making these vents, they should be clear of any glass, panes, screens, and curtains that may hamper the exit of gases. Also, if attack companies notice turbulence in the structure, then additional exhaust openings should be made to relieve the pressure. PPV should be delayed if there are backdraft conditions, if application of water is difficult, if the construction of the house is balloon frame, or if there is concealed fire in the walls. As far as fire dynamics and PPV, the author states that fire within the path of air will be pulled toward the exhaust, fire not within the path of air will be pulled toward the vent at a speed relative to its proximity to the vent, and if no exhaust vent exists, the fire will spread back to the inlet opening. Also, a hose stream should never be directed into a vent opening, for fear of pushing fire onto attack companies.\cite{}\\

	\indent This document presents a plan for the implementation of a positive pressure attack strategy for fire departments in North America. The document discusses training strategies, fan technology selection, and safety issues. The author recommends a three-pronged approach, with fans first being introduced after suppression, then, after crews are comfortable with that tactic, fans should be introduced once the fire is controlled, and, after that, they should be used prior to fire attack. Crews should practice safety measures compliant with NFPA 1500. Also, exposures should be immediately identified and addressed before the fan is turned on. The author advocates the use of a conventional blower unit as opposed to a turbo unit, because the turbo unit does not “seal” the doorway, allowing additional air movement around the door frame, giving the fire the opportunity to “burn back.”  The author also discusses different types of powered fans (gasoline, electric, water driven etc.), advocating that the fan should also be easy to use and durable.\cite{ImplementStrat}\\

	\indent Garcia begins by outlining the changes in home construction and furnishing in the past half-century. Fires flash over in less than a third of the time they did with traditional furnishing. He maintains that the flammability and toxicity of today’s fire gases make it more essential than ever before for effective ventilation of building fires. He cites the CDC in describing the products of combustion of modern fires as “a soup of carcinogens.” Coupled with fuel loads with higher potential heat release rates, the emphasis on lightweight construction in modern buildings make roof ventilation operations more dangerous than before. The solution, therefore, is a coordinated positive pressure attack, implemented after proper training and with adequate command and control procedures. He also says that PPV is not as simple as it was in previous years, because of the risk of entraining additional oxygen to these ventilation-limited fires. For this reason, departments should use a phased approach to PPA implementation, only using it department wide when proper training and operating procedures have been established.\cite{GarciaPPVatStructures}\\

	\indent In a letter to the editor of Fire Chief Magazine, Routley, a fire protection engineer, expresses his concern that Garcia and Kauffman understated some grave hazards in their positive pressure attack strategy. He says that it is incorrect to say that PPV will not push fire throughout a building. If PPV is used incorrectly, it certainly could push fire into void spaces or rooms remote from the fire. He also maintains that their stance that you can’t make a vent hole in the wrong location is a serious understatement, because some vent hole locations will spread fire to previously uninvolved areas of the structure. He also criticizes their lack of concern regarding the use of PPA under backdraft conditions and with basement and attic fires. Garcia and Kauffman address each of his points separately, saying that their tests, using buildings constructed with dimensional lumber and gypsum board and having a fuel load of pallets and straw, have shown that if an adequate exhaust is provided, the fire will only spread towards that vent opening, due to the negative pressure differential. \cite{PositiveOversight}\\

	\indent In this article, Garcia and Kauffmann state that while PPA is a useful and effective tactic, certain concepts must be clearly understood before it is put into use. They cite a recent LODD in Massachusetts, where crews arrived to find a basement fire. A chief reported that the fire had been knocked down, and requested ventilation. PPV was started at the front door, and soon after a large cloud of thick black smoke was seen. The chief radioed a mayday for all personnel to exit the building, and one firefighter died due to accelerated fire spread. The authors outline 4 hazards to be considered with the use of PPA. The first is that PPA should not be used when victims or firefighters are standing at exhaust openings. The purpose of PPV is to create a high-pressure area of clean air on the entry side of the fire, and to blow hot gases and smoke out of the exhaust vent. If a person is standing at that vent, they will be right in the path of these gases. The second matter that they highlight is that PPA should never be started when there are crews inside of the building, and if there are crews inside of the building, then they must back out before PPA is initiated. This is because the flow path that the positive pressure attack will create may not be known, or the exact location of fire companies may not be known, or both. Starting PPA with crews inside may place those crews between the fire and the exhaust point. A third consideration is that PPA should not be used in flammable atmospheres, because the use of PPV may force these gases or particles to an ignition source. Their fourth caution was that PPA should not be used in backdraft scenarios, where there is a severely underventilated fire with high heat conditions. They go on to say that if backdraft conditions are present, there is a very slim likelihood of viable victims. The authors also briefly touch on other points as well. They maintain that a comprehensive training program must be in effect before any PPA strategy is implemented. They also emphasize the importance of an adequate exhaust opening. They give no particular size, but they say that if the exhaust opening is inadequate, heat and smoke will just circulate, creating a condition similar to a convection oven. Finally, caution should be used during the overhaul phase, as PPV’s efficiency at removing smoke can hide it just as efficiently when crews are looking for hidden fires.\cite{GarciaPPVatStructures}\\

	\indent Positive Pressure attack is a tactic whose name can be deceptive. It must be thought about in terms of negative pressure, not positive pressure. Fire naturally moves from areas of high pressure to areas of lower pressure. According to the author, the two main factors that influence the high pressure in a fire compartment are the pressure created by the fire itself and the increase in pressure cause by water application. When an exhaust vent is made and PPV is applied, the fire will quickly seek the area of lowest pressure, which will be out of that exhaust vent. The author also addresses several concerns that should be kept in mind when considering PPA, including insufficient exhaust, backdraft conditions, and explosive environments. Also, with regards to fire spread in attic and void spaces, gable vents and similar natural ventilation openings in the attic do not provide a sufficient negative pressure zone for the fire to spread to these areas. If a vertical vent hole were cut, however, a large negative pressure zone would be created, allowing the fire to spread into the attic area, which may negatively impact the operation. For this reason, the most preferred exhaust location should be horizontal, such as a failed window, and not vertical. When using positive pressure attack, firefighters much think of it as channeling fire into the largest negative pressure zone, and not as much in terms of the positive pressure created by the fan.\cite{NegativeThinkingGarcia}\\

	\indent After a close call where a firefighter went through the roof, an Arizona fire department decide to reevaluate its ventilation strategies. After looking back on incidents where ventilation was implemented, it was determined that the strategies that had been used had been ineffective, or even counterproductive. They determined that the vertical ventilation strategy that they had been using was hard to coordinate and often resulted in a blindly cut vent hole or a vent hole that was not cut in conjunction with the fire attack company. The department eventually decided in order to get what they wanted out of ventilation; a positive pressure attack strategy was a better option. PPA improves interior conditions, which makes a safer operating condition for firefighters, and enables search companies to find victims more safely and effectively. Acknowledging that the fire service is deeply rooted in tradition, the author states that it is sometimes necessary to depart from “the way we’ve always done it” and find strategies and methods that are safer and more effective for firefighters.\cite{OpenEyesPPA}\\

	\indent A battalion chief in the Sacramento, CA, area describes a fire where he was the “roof sector command.” The fire occurred in a structure that used to be the library of a community college, but had been converted to the student service center. The chief was on the roof with two truck companies when he observed heavy black smoke coming from every hole in the roof. He immediately radioed the IC for a third and fourth alarm, when suddenly there was a large explosion, and a large part of the roof was burned away. The crews evacuated the roof and a defensive operation was commenced. It was later discovered that a firefighter on the ground had observed decreased smoke conditions in the interior of the structure, and had started a PPV fan at the door. The additional air entrained into the backdraft conditions in the attic caused the smoke to ignite and resulted in a violent fire development. The author quotes John Mittendorf in saying, “Never positive pressure a building until you have first evaluated the attic for backdraft potential.” The author expands this idea to include not only the attic, but all areas of a burning building. PPA and PPV can be useful and safe operations in a fire attack, but they must be used correctly, and in the right place at the right time.\cite{UnderstandPPAChallenge}\\

	\indent Tactical ventilation can often be a manpower-consuming process. The author of this article, a fire Captain on a department in Canada, undertook the task of addressing the fire ground needs of a smaller volunteer fire department with three to four man staffing and additional responding companies ten to twenty minutes away. The author examined the merits of a positive pressure attack with exterior suppression through the exhaust opening. The tests occurred in a concrete burn building with a fuel loading of pallets and straw. The fire was allowed to grow until ceiling temperatures in the burn room were greater than 1,000oF. Then, the fire room window and front door were opened and PPV was started. Temperatures and visibility in the hallway leading up to the fire room drastically improved. After that, a straight stream was applied to the ceiling of the fire room for 10 seconds through the exhaust opening. After the 10 second indirect attack, water was directly applied to the seat of the fire. The PPV prevented flames, gases, and smoke from burning back into the corridor throughout the entire process. Positive pressure attack in conjunction with an exterior application of water is an effective method to quickly extinguish a fire and clear the atmosphere for fire departments with limited initial fire ground resources.\cite{CambellExtfire}\\

	\indent Garcia and Kauffmann outline the “Big Threee” precautions of a positive pressure attack. These are the three main problems that cause rapid fire development or other dangerous fire ground conditions. The first is Exhaust. Adequate exhaust must be provided so that the fire will not burn back towards the entryway that firefighters are using to attack the fire. Some critics say that too much exhaust will decrease pressure so much that nuisance smoke will not be able to be exhausted from the structure. The authors maintain that the nuisance smoke is not as much of an immediate concern as the IDLH conditions that may present themselves if the vent hole is inadequate. The method that crews should use to evaluate the exhaust adequacy is to position the fan so that there is a small neutral pressure zone at the top of the doorway. If this zone is filled with smoke and flames, exhaust is inadequate. If the conditions in this space are improving, then the PPA is working as intended. The second consideration is Entry. This precaution states that fans should never be placed into operation after attack crews have made entry. If firefighters have made entry in low visibility conditions, and then a fan is placed into operation behind them, they may have passed potential hazards for the simple reason that they could not see them. These hazards could be increased by the application of PPV and could trap the firefighters. The third precaution that is outlined is Execution. Firefighters and officers using a positive pressure attack must be fully trained and understand the possible outcomes of their actions. The training must be more comprehensive than training on the operation of the blowers. A lack of understanding of positive pressure attack can lead to an uncoordinated fire attack and can cause the deterioration of fire conditions. Users of PPA must understand the situations where it cannot be used (Backdraft, imminent rescue, explosive atmospheres, high wind conditions, etc.). The authors sum up by saying that PPA and PPV are so effective that even when they are used improperly, a positive result is often achieved. If departments look out for “The Big Threee,” they will be able to conduct safe and effective ventilation operations on the fire ground.\cite{BigThreee}\\

	\section{Training Manuals}

	\indent John Mittendorf outlines Ten Commandments for firefighters to follow regarding the use of positive pressure attack on the fire ground. 1- Have a goal and know how to accomplish it 2- Have charged lines in place 3- Determine the location of the fire and the potential airflow route 4- Make the exhaust opening first 5- Seal the door with a cone of air 6- Do not block the entrance opening 7- Use the correct size and number of blowers for each application 8- Ensure the proper exhaust opening size 9- When appropriate, use sequential ventilation 10- Consider the carbon monoxide exhaust when using gasoline blowers. Mittendorf emphasizes the need for control of the airflow between the inlet and exhaust openings. Any extra airflow will decrease the efficiency of the PPV operation. He also recommends that the exhaust opening’s size should depend on the size and output of the fan. For a smaller fan, an exhaust ¾ the size of the inlet opening should be used. For multiple fans moving a very large volume of air, an exhaust opening that is 1 ¾ times the size of the entrance opening may be necessary. A good test to judge whether or not the exhaust opening is adequate is to be alert for the presence of an exhaust odor from the fan. If exhaust can be smelled, a larger exhaust hole must be made. Mittendorf also stresses the need to create an exhaust opening before pressurization is started. One of the reasons he underlines this is when the fire building is balloon-frame construction. Pressurizing the structure can force fire into the walls and from there into the attic. Other times when PPA should not be used include attic fires, VES scenarios, incidents where the location of the fire is unknown, and when the location of interior crews cannot be determined. Additionally, Mittendorf adds that PPV should not push fire throughout a structure, since air velocities throughout the structure are relatively low, with most of the air velocity being seen at the entrance and exit openings. With regards to victims being located between the seat of the fire and the exhaust opening, he maintains that victims are more likely to be killed by smoke inhalation than by thermal burns, and any victims that are found this close to the fire are likely to be untenable anyways.\cite{TruckCoOps}\\

	\indent The 6th edition of Essentials of Firefighting and Fire Department Operations mentions that positive pressure can be used in a pre-attack method to clear heat and smoke, cool interior temperatures, and improve visibility conditions inside of a building.\cite{IFSTAESS}\\

	\indent Fire Engineering’s Handbook for Firefighter I and II highlights some basic considerations to have in mind when implementing a positive pressure attack. Chief among these considerations are to vent the exhaust opening before pressurization has started and to control the flow of air along the path between the entrance and exhaust openings. It warns readers that extra caution must be taken with the use of this tactic, because, while it can clear visibility and smoke conditions for victims and firefighters, it can also supply air to the fire, precipitating rapid fire growth, and possibly drive fire into void spaces. This raid fire development could make conditions worse for victims in between the fire and exhaust as well as for firefighters whose whereabouts in the structure are unknown.\cite{FE_FireI_II}\\

	\indent In John Norman’s Fire Officer’s Handbook of Tactics, he makes little mention of positive pressure ventilation used in an attack role. He recommends PPV for low-heat, high-smoke fires, such as a smoldering chair or mattress. Norman recommends having a charged attack line in place at all times, lest a smoldering fire be intensified due to increased air entrainment. Care must also be taken if the fire is suspected to be located in concealed spaces, as PPV could cause further extension of the fire.\cite{NormanHandbook}\\

	\indent  The 2nd edition of Fundamentals of Fire Fighter Skills contains a short blurb about positive pressure attack. It describes PPA as a method to reduce toxic smoke and temperatures in the path from the attack line to the fire. The textbook lists several important considerations when conducting PPA. First, the fire building must be adequately sealed for pressurization to work correctly. If there are too many openings, there may be inadequate pressure for the flow path to be effective. Also, the exhaust opening should be approximately the same size as the entrance opening. This will allow positive pressure within the structure while allowing enough smoke and heat to be exhausted so that these products of combustion will not be recirculating within the structure. To increase efficiency, search companies should close doors to uninvolved portions of the structure, optimizing the air flow between inlet and exhaust. Caution should always be excercised when considering a positive pressure attack, because increased air entrainment can result in increased fire spread, and can cause hidden fires to flare up again.\cite{NFPA_IAFC}\\

 	\indent The 3rd edition of the Firefighter’s Handbook Essentials of Fire Fighting divides the use of PPV into two areas: post-fire attack PPV and positive pressure attack. The text states that while PPA has been used successfully by many fire departments, the rapid development and high heat release of modern fires have limited its applicability on the modern fire ground. When used correctly, PPA can reduce smoke and heat conditions in the path from the front door to the fire. It should not be used, however, when there is an imminent rescue of a civilian or a firefighter, if there are firefighters in the structure before the fan is started, if the location of the fire is unknown, if there is not an adequate exhaust location, or id the structure shows signs of overpressurization or backdraft. Additionally, PPA can be used for basement fires, provided that an adequate exhaust is provided.\cite{FFHandbook}\\
	
%	\bibliography{Regan_Report_bib}
%	\bibliographystyle{mla}
\end{document}